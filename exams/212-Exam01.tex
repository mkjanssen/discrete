\documentclass[11pt,answers,addpoints]{exam}
\usepackage{geometry} 
\usepackage{amsmath,amsthm,amssymb,amsfonts,parskip,xypic}
\geometry{letterpaper}
\usepackage{graphicx}
\usepackage{color}
\usepackage{amssymb}
\usepackage{multicol}
\usepackage{epstopdf}
\DeclareGraphicsRule{.tif}{png}{.png}{`convert #1 `dirname #1`/`basename #1 .tif`.png}
\usepackage{tikz}
\usepackage{verbatim}

%\def\le{\mkern3mu\mathrel{\vrule height 1.5ex\joinrel
 % \mkern-3mu\raise.2ex\hbox{$=$}\joinrel
  %\mkern-4mu\vrule height 1.5ex}\mkern3mu}
%\def\li{\mkern3mu\mathrel{\vrule height 1.5ex\joinrel
%  \mkern-3mu\raise.2ex\hbox{$=$}}\mkern3mu}

\usepackage{coordsys}


\usetikzlibrary{positioning}

\newcommand{\bipgraph}[2]{%
    \begin{tikzpicture}[every node/.style={circle,draw}]
    \foreach \xitem in {1,2,...,#1}
    {%
    % first set
    \node at (0,\xitem) (a\xitem) {};
    % second set
    \node at (2,\xitem) (b\xitem) {};   
    }%

    % connections
    \foreach \x [count=\xi] in {#2}
    {% 
    \foreach \tritem in \x % <-- Here no braces to make it a foreach list also not \xi but \x
    \draw(a\xi) -- (b\tritem);
    }
    \end{tikzpicture}  
}


%\pagestyle{empty}
%\setlength{\oddsidemargin}{.05in}
%\setlength{\evensidemargin}{-.50in}
%\setlength{\textwidth}{7in}
%\setlength{\topmargin}{-.250in}
%\setlength{\headheight}{0in}
%\setlength{\headsep}{.25in}
%\setlength{\topskip}{0in}
%\setlength{\textheight}{9.05in}
%\setlength{\parindent}{0in}
%\font\bigbf = cmbx10 scaled \magstep1
%\font\medbf = cmbx10 scaled \magstephalf



\newcommand{\class}{Math 212}
\newcommand{\term}{Spring 2021}
\newcommand{\examnum}{Exam 1}
\newcommand{\examdate}{Mar.~5, 2021}
\newcommand{\timelimit}{50 minutes}



%\shadedsolutions


% Include answers?
\noprintanswers
%\printanswers

%\bracketedpoints


% Where should the points be?
% Default
%\nopointsinmargin
% Left margin
\pointsinmargin
\marginpointname{ \points}
% Right margin
%\pointsinrightmargin






% Nice way to TeX sets
\def\set#1{\left\{ {#1} \right\}}
\def\setof#1#2{{\left\{#1\,|\,#2\right\}}}

\def\i{{\bold i}}

\def\j{{\bold j}}

\def\k{{\bold k}}

%\def\vec{u}{{\bold u}}

%\def\vec{v}{{\bold v}}

\def\w{{\bold w}}

\def\n{{\bold n}}

\def\d{{\partial}}


\def\C{{\mathbb C}}
\def\Z{{\mathbb Z}}
\def\F{{\mathbb F}}
\def\bF{{\mathbb F}}
\def\Q{{\mathbb Q}}
\def\R{{\mathbb R}}
\def\P{{\mathbb P}}
\def\N{{\mathbb N}}





\begin{document}

\runningheader{\bfseries Math ZZZ}{}{\bfseries Quiz \#X (Continued)} 



\header{\bfseries\large \class}%\\Michael Janssen} 
{\bfseries\large \examnum} 
{\bfseries\large \examdate} 


%Name: \makebox[4.15in]{\hrulefill} Section: \makebox[0.75in]{\hrulefill} % Section:\enspace\hrulefill}
\makebox[\textwidth]{Name:\enspace\hrulefill}

%\vec{v}space{0.3in} 
%\makebox[\textwidth]{Section:\enspace\hrulefill}

\vspace{0.2in}

\textbf{Instructions: } This exam has \numquestions\ questions for a total of \numpoints\ points. Answer each question as completely and clearly as you can. Make sure to use complete sentences. Take care to not oversimplify any problems. You have \timelimit.
%Your responses do not need to be typed, but you are encouraged to do so.

%\textbf{The only resources you are allowed} are your notes and discussions with your professor (both of which you are encouraged to make use of)! No discussions with your peers, tutors, consultation of other books, websites, etc., are allowed. If there is any reason to believe you have consulted with a forbidden source, you will earn a zero on the take-home component of this exam and the matter will be reported to the Student Life Committee.
%Answer each question completely on the sheets provided. Take care to not oversimplify each problem.

\vspace{2in}

\begin{center} 
\gradetable[v]%[pages] 
\end{center}

\hfill

\newpage





\medskip

\begin{questions}







\question[10] Some hungry students head to the Healthy Snack Shop where they can choose one of five kinds of fruit, one of three herbal teas, and one of six flavors of wraps to get packed in a box. How many possible snack boxes are there?

\vfill

\question[15] One college sent another a report saying that 119 students took Calculus I in a Fall semester. The report notes that during the next term, 96 of these studnets took Calculus II, 53 took Discrete Mathematics, and 39 took Physics II. The report says that 48 of the students took Calculus II and Discrete Mathematics, 31 of the students took both Discrete Mathematics and Physics II, 32 of the students took both Calculus II and Physics II, and 22 of the students took all three courses. We examine the report and sense an error is present. Why?

\begin{solution}

\end{solution}

\vfill

\newpage

\question[8] Let $f : \set{1,2,3,4,5} \to \set{a,b,c,d}$ be a function. Justify your answers to the following questions.

\begin{parts}
\part How many such functions $f$ exist?

\vfill 

\part How many such functions are injective?

\vfill 


\part How many such functions are surjective?

\vfill 


\part How many such functions are bijective?

\vfill 
\end{parts}

\newpage

\question[10] Let $S = \set{a,b,c,d}$, $T= \set{1,2,3}$, and $U = \set{b,2}$. Which of the following statements is true? Which is false? Explain.

\begin{parts}
\part $\set{a}\in S$

\begin{solution}[1in]
\end{solution}

\part $\set{a,c,2,3}\subseteq S\cup T$

\begin{solution}[1in]
\end{solution}

\part $U\in \mathcal{P}(S\cup T)$

\begin{solution}[1in]
\end{solution}

\part $\emptyset \subseteq \mathcal{P}(S)$

\begin{solution}[1in]
\end{solution}

\part $\set{\emptyset} \subseteq \mathcal{P}(S)$

\begin{solution}[1in]
\end{solution}






\end{parts}

\clearpage

\question[8] Dr. Janssen wishes to distribute 10 Rubik's cubes to the 19 students in Math 212. \textbf{In answering the following questions, you only need to give an expression that evaluates to the correct answer; the actual number is not required for full credit.}  %Consider the relation $<$ (less than) on the set of integers $\Z$. Is $<$ an equivalence relation? Justify your answer.

\begin{parts}
\part Suppose first that the cubes are identical, so it would be silly to give anyone more than one cube. How many ways are there to accomplish this? 

\vfill


\part Now suppose each cube is a different model, so that it's not unreasonable to give more than one cube to a particular person. Still, we'll ensure that no one gets more than three cubes. How many ways are there to accomplish this?

\vfill
\end{parts}






\end{questions}









\end{document}  