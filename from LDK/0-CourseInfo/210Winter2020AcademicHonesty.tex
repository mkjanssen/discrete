\documentclass[12pt]{article}
\usepackage[margin=.4in]{geometry}
\usepackage{nopageno}
\usepackage{hyperref}
\usepackage{multicol}
\usepackage{graphicx}
\usepackage{enumitem}

\def\signed #1{{\leavevmode\unskip\nobreak\hfil\penalty50\hskip2em
  \hbox{}\nobreak\hfil(#1)%
  \parfillskip=0pt \finalhyphendemerits=0 \endgraf}}

\newsavebox\mybox
\newenvironment{aquote}[1]
  {\savebox\mybox{#1}\begin{quote}}
  {\signed{\usebox\mybox}\end{quote}}
  
%To do: Evaluation percentages, hyperlinks

\begin{document}
\vspace{-1.2in}
\begin{center} \textbf{\Large{Academic Honesty in Mathematics}} \\
Math 210 Section 01, Winter 2020
\end{center}

\noindent The following document serves as a guide for you to understand what appropriate collaboration looks like in this class.  You should read this document carefully as well as Section 4 of the student code \url{http://www.gvsu.edu/studentcode/}.  I understand that you as students face many stresses and temptations.  I also understand that standards for collaboration may differ in different courses and disciplines. However, academic dishonesty will not be tolerated. I hope that after reading this document you better understand the expectations in this course. If at any point you have any question about academic honesty please let Professor Keough know.  In issues of academic honesty it is much better to ask for permission than beg for forgiveness. 

\vspace{-.2in}
\section*{First Things}

\textbf{Collaboration is important.}  Working together can contribute to deeper understanding and more growth in mathematical skills.  Talking about the material can help us get past points where we are stuck and identify points where we haven't thought deeply enough.  Even at the highest levels of mathematical research it is becoming rare to see research papers that have only one author.  I believe the mathematical community recognizes that together we can go further and I believe that you can go further in your development with collaboration as well. \\

\noindent\textbf{Individual accountability is also important.}  Especially in this course, it is important that you have a deep individual understanding of the content.  This course prepares you for many of the courses in the math major/minor and future professors will expect that you have certain skills.  It is also a truth of academia that at the end of the semester you will be assigned a grade based on your individual understanding of course material and your individual ability to apply it. \\


\noindent\textbf{What does collaboration look like in mathematics?}  Collaboration looks like two or more people who are \emph{at the same stage of the problem solving process} sitting together talking about a problem.  Typically (and I highly recommend that) some individual thinking happens before collaboration begins. That is, you should look over the problem, make sure you know the definitions of all terms in the problem, and think about how you would attempt the problem before collaborating.  Collaboration does not look like lecture - i.e., it should not be one person telling another person.  The group should move together meaning that one or more members of the group should not move on if one or more members of the group do not understand.  Collaboration in this course, perhaps unlike in other courses/disciplines, does not look like ``divide and conquer".  That is, you should each do each problem (not divide up the work and then explain to each other). Ultimately it is your responsibility to make sure you are using collaboration to go further and deeper and not to get out of the process.  \\

\noindent\textbf{How do mathematicians use resources?}  While Google is an extremely important tool at the research level of mathematics, you can not in this class use resources besides our textbook, course notes, GVSU's MTH 210 YouTube videos, and anything posted on Blackboard.  When learning to prove you will prove things that many people who have gone before you have already proved.  Under no circumstances should you look up a proof or solution on the internet.  This course is about process more than product - you won't necessarily remember how a specific proof goes, but you should learn many things about the process of writing a proof.  Looking up a proof on the internet skips the important process part.\\

\noindent\textbf{What is academic dishonesty?}  There are many forms of academic dishonesty, but perhaps the most common one I see in this course is the act of submitting the work of someone else (either a peer or from an outside resource) as if it were your own.  This misleads the instructor in that the instructor thinks you have learned and understood course content when you have not.




\section*{Assignments in this course}

Here I will describe expectations for each of the assignments for this course. It is considered academic dishonesty to not follow the outlined rules.  Remember that I am available to help you on any assignment for this course.  I will not give you a solution (this skips the important process part), but I will do my best to help you overcome your difficulties.

\begin{enumerate}
\item \emph{Proof Portfolio:}  On all problems in the proof portfolio no outside resources are allowed.  No collaboration with anyone besides Dr. Keough is allowed. No other students, tutors, or other faculty may help you.

\item \emph{Skill Mastery Quizzes:}  Preparation for these together is acceptable and encouraged, but no collaboration is allowed during quizzes.

\item \emph{Exams (Midterms and Final):} Preparation for these together is acceptable and encouraged. Exams are closed book, closed notes, and closed classmates - any use of resources on the exam is considered academic dishonesty. 


\item \emph{Synthesis Activities:}  Collaboration is allowed and encouraged. See the collaboration section on the first page to be clear about what this looks like. Submitting anything you do not understand or copying from any resource (such as an online resource or a classmate) will be considered academic dishonesty. In the event that you are asked to type these, sharing of any electronic files is not allowed.


\item \emph{Preview Activities:}  You may use any resources you?d like and collaborate with anyone you wish on these assignments. Submitting anything you do not understand or copying from any resource will be considered academic dishonesty.





\end{enumerate}

\section*{Consequences of Academic Dishonesty}

Since the entire educational system relies on your academic honesty any instance of academic dishonesty will not be tolerated.  Evidence of dishonest behavior will be grounds for not earning credit on the assignment or, depending on severity, failure of the course.  In addition, instances of academic dishonesty will be reported through GVSU's central system in the Dean of Students office.  In all cases the guidelines established in the GVSU catalog and the GVSU student code will be followed.  Being complicit in situations of academic dishonesty (such as sharing your completed assignment)  will be dealt with in the same way.\\




\noindent I expect that we will have no issues of academic dishonesty if you follow the guidelines above and ask me about any issues that come up.  Remember, again, that \emph{I want you to succeed} which means I want you to grow in your mathematical ability.  Be informed by knowing course policies. Avoid temptations by starting assignments early and being a frequent visitor to office hours.


\section*{If you are tempted}

Each of you may feel tempted to break one of our rules for academic honesty at some point in the semester. The work can be difficult, and many of you are under a lot of stress. If you are considering academic dishonesty, please STOP, take a breath, and understand two things: 
\begin{itemize}
\item I want you to succeed in the course, and you have fair and honest ways of getting help, and many opportunities for revision of work. There is no need to be academically dishonest! Just do your best and revise later.
\item Most students who commit academic dishonesty are caught --- you'd be surprised how easy it is to detect --- and the consequences are severe. 
\end{itemize}


\end{document}
