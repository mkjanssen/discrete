\documentclass[10pt]{article}
\usepackage[margin=1in]{geometry}
\usepackage{nopageno}
\usepackage{hyperref}
\usepackage{multicol}


\usepackage{amsmath,amssymb,graphicx}

\newcommand{\R}{\mathbb{R}}
\newcommand{\Z}{\mathbb{Z}}
\newcommand{\E}{\mathbb{E}}
\newcommand{\Q}{\mathbb{Q}}
\newcommand{\cM}{\mathcal{M}}
\usepackage{xcolor}
\newcommand{\blue}{\textcolor{blue}}

\usepackage{versions}
\excludeversion{solution}

\newcommand{\bs}{\begin{solution}}
\newcommand{\es}{\end{solution}}



\begin{document}
\vspace{-1.2in}
\begin{center} \textbf{\Large{Skill Mastery Quiz 2}} \\
Communicating in Math (MTH 210-01)\\
Winter 2020
\end{center}



\noindent Name: 




\begin{itemize}
	\item[L1-2] Consider the following (true) conditional statement: 
	\begin{center}
	If the function $f$ is continuous at $a$, then $\displaystyle{\lim_{x\to a} f(x)}$ exists.
\end{center}
	Identify the hypothesis and conclusion of this conditional statement.
	
	\bs\textcolor{blue}{The hypothesis is ``the function $f$ is continuous at $a$" and the conclusion is ``$\displaystyle{\lim_{x\to a} f(x)}$ exists".}\end{solution}
	\vspace{.5in}
	
	
Assume the above conditional statement is true. Assuming \emph{only} the conditional statement and that a function $f$ is not continuous at $7$, what can you conclude (if anything)?  Explain your answer.

	\bs\textcolor{blue}{Given that the function $f$ is not continuous at $7$ we know the hypothesis of the conditional statement is false. Thus we cannot conclude anything, since the statement makes no promises about what happens if a function is not continuous at a given $a$. See Quiz 1 solutions for an explanation with a truth table.}\end{solution}

\vfill



\item[L2-1] State the definition of odd integer precisely:

An integer $n$ is odd provided that...

\bs\textcolor{blue}{ there exists an integer $q$ such that $n=2q+1$.}\end{solution}

\vfill
 Then outline a proof that if $x$ is odd and $y$ is even then $xy$ is even. (Make sure to include key details - like what things are integers.)
 
 \bs\textcolor{blue}{Suppose $x$ is odd and $y$ is even. Then there exist integers $a$ and $b$ such that $x=2a+1$ and $y=2b$. Then $xy=(2a+1)(2b) = 4ab+2b$ by substitution and algebra. By the distributive property $x+y=2(2ab+b)$. Let $q=2ab+b$. Note that $q$ is an integer because $a$ and $b$ are integers and the integers are closed under addition. Then $x+y=2q$ for the integer $q$ and so $x+y$ is an even integer.}\end{solution}
\vfill
\vfill
\vfill
\vfill

\newpage

\item[L3-1] Construct a truth table for $(\neg P\vee Q) \rightarrow R$.

\bs
\begin{center}
\begin{tabular}{c|c|c|c|c|c}
$P$ &$Q$ &$R$ &$\neg P$ &$\neg P \vee Q$ &$(\neg P \vee Q)\rightarrow R$\\
\hline
T &T &T  &F &T &T\\
T &F &T  &F &F &T\\
F &T &T  &T &T &T\\
F &F &T  &T &T &T\\
T &T &F &F &T &F\\
T &F &F &F &F &T\\
F &T &F &T &T &F\\
F &F &F &T &T &F\\
\end{tabular}
\end{center}
\end{solution}
\vfill
\vfill
\vfill


\item[L4-1]  Write the set $\left\{ \sqrt{2}, \left(\sqrt{2}\right)^3, \left(\sqrt{2}\right)^5,\dots\right\}$ in set builder notation.

\bs \textcolor{blue}{One way to do this is write $\{x\in \mathbb{R} \mid x= (\sqrt{2})^n \text{ for some odd natural number } n\}$. Another way is $\{x\in \mathbb{R} \mid x= (\sqrt{2})^{2n-1} \text{ for some } n\in\mathbb{N}\}$. There are several other ways one could correctly describe this set.}
\end{solution}

\vfill
\end{itemize}
	

\end{document}
