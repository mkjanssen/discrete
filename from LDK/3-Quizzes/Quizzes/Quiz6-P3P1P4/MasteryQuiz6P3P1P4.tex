\documentclass[10pt]{article}
\usepackage[margin=1in]{geometry}
\usepackage{nopageno}
\usepackage{hyperref}
\usepackage{multicol}


\usepackage{amsmath,amssymb,graphicx,amsthm}

\newcommand{\R}{\mathbb{R}}
\newcommand{\Z}{\mathbb{Z}}
\newcommand{\E}{\mathbb{E}}
\newcommand{\Q}{\mathbb{Q}}
\newcommand{\N}{\mathbb{N}}
\newcommand{\cM}{\mathcal{M}}
\usepackage{xcolor}
\newcommand{\blue}{\textcolor{blue}}

\usepackage{versions}
\excludeversion{solution}

\newcommand{\bs}{\begin{solution}}
\newcommand{\es}{\end{solution}}

\newtheorem{thm}{Theorem}


\begin{document}
\vspace{-1.2in}
\begin{center} \textbf{\Large{Skill Mastery Quiz 6}} \\
Communicating in Math (MTH 210-01)\\
Winter 2020
\end{center}



\noindent Name: 




\begin{itemize}
	


\item[P3-3]  The following statement is incorrect:
		\begin{center}
For each integer $n$, if $n$ is odd, then $(n^2+1)$ is a prime number.	
	\end{center}
			Show the statement is false using a counterexample. You should clearly explain why the counterexample you found shows the statement is false. (If you don't remember what a prime number is just ask!)


\bs \textcolor{blue}{This statement is false and there are many counterexamples. As one, let $n=3$. Then $n\in\Z$ and $n$ is odd (since $3=2(1)+1$ and $1\in\Z$). Additionally, $n^2+1=3^2+1 = 10$ which is not prime (since $10=5\cdot 2$). Thus we have found an integer for which the hypothesis is true and the conclusion is false, making the statement false.}\end{solution}
\vfill



\item[P1-2] Consider the following statement:
		\begin{center}
		For all natural numbers $p$ and $q$, if $p$ and $q$ are twin primes other than $3$ and $5$ , then $pq+1$ is a perfect square and $36$ divides $pq+1$.
		\end{center}
	State what you would assume in a direct proof. 
	
	\bs\textcolor{blue}{Assume that $p$ and $q$ are natural numbers and $p$ and $q$ are twin primes other than $3$ and $5$. (Basically, we assume the hypothesis.)}\end{solution}
	
	\vfill
	
	
	State what you would assume in a proof by contradiction.
	
	\bs\textcolor{blue}{Assume that there exist natural numbers $p$ and $q$ such that $p$ and $q$ are twin primes other than $3$ and $5$ and that $pq+1$ is not a perfect square or $36$ does not divide $pq+1$. }\end{solution}
\vfill

\newpage



\item [P4-1]  Consider the following proposition and proof. Is the proof correct? If not, explain any major mathematical errors. If so, does the proof meet our writing guidelines? 
	
		\begin{thm}
		If $a$ is an odd integer then $3a+2$ is an odd integer.
		\end{thm}
		\begin{proof}
		We will use a direct proof. For $3a+2$ to be an odd integer there must exist an integer $n$ such that
			\[3a+2 = 2n+1.\]
		By subtracting $2$ from both sides of this equation we obtain
			\begin{align*}
			3a &= 2n-1\\
			&=2(n-1)+1.
			\end{align*}
		By the closure properties of integers, $n-1$ is an integer, and hence, the last equation implies that $a$ is an odd integer. This proves that if $a$ is an odd integer then $3a+2$ is an odd integer.
		\end{proof}

\bs \textcolor{blue}{Though the statement is true, the proof is wrong for multiple reasons. For one, they start by assuming the conclusion (by saying there exists an integer $n$ such that $3a+2=2n+1$. Moreover, they only show then that $3a$ is odd, not that $a$ is odd (though showing $a$ is odd so that is the hypothesis anyway, which isn't what they should be trying to show!)} \end{solution}

\end{itemize}
	

\end{document}
