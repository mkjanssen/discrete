\documentclass[10pt]{article}
\usepackage[margin=1in]{geometry}
\usepackage{nopageno}
\usepackage{hyperref}
\usepackage{multicol}


\usepackage{amsmath,amssymb,graphicx}

\newcommand{\R}{\mathbb{R}}
\newcommand{\Z}{\mathbb{Z}}
\newcommand{\E}{\mathbb{E}}
\newcommand{\Q}{\mathbb{Q}}
\newcommand{\cM}{\mathcal{M}}
\usepackage{xcolor}
\newcommand{\blue}{\textcolor{blue}}

\usepackage{versions}
\includeversion{solution}

\newcommand{\bs}{\begin{solution}}
\newcommand{\es}{\end{solution}}



\begin{document}
\vspace{-1.2in}
\begin{center} \textbf{\Large{Skill Mastery Quiz - Omar L3, L4, P1}} \\
Communicating in Math (MTH 210-01)\\
Winter 2020
\end{center}






\begin{itemize}
	

\item[L3-token] 	Construct a truth table for $P\vee (Q\rightarrow R)$.


\vfill
\vfill

\item[L4-token] Write the set $\{\dots, -1, 2, 5, 8, 11,\dots\}$ in set builder notation.
\vspace{2in}

\vfill
\newpage

\item [P1-token] Consider the following theorem:
	\begin{center}
	For all integers $a,b,$ and $c$, if $a\mid b$ and $b\mid c$ then $a\mid c$.
	\end{center}
State what you would assume in a direct proof of theorem.



\vfill
State what you would assume in a proof by contradiction of the theorem.
 

\vfill
\end{itemize}
	

\end{document}
