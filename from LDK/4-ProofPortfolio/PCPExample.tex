\documentclass{article}  %Need this.

\usepackage{amsmath,amsthm,amssymb}


\newtheorem*{thm}{Theorem}
\newtheorem*{cnj}{Conjecture}
\newtheorem*{lem}{Lemma}
\newtheorem*{cor}{Corollary}
\newtheorem*{prop}{Proposition}

\newcommand{\N}{\mathbb{N}}
\newcommand{\Z}{\mathbb{Z}}
\newcommand{\R}{\mathbb{R}}





\title{Proof Portfolio Problem \#  1--  Example--}
\author{--Dr. Keough --}
\date{}

\begin{document}
\maketitle  %This will add the title, author, and date located above


Here's the problem: You work for a cell phone company which has just invented a new cell phone protector and wants to advertise that it can be dropped from the $n^{th}$ floor without breaking. 
If you are given 2 phones and a 100 story building, how do you guarantee you know the highest floor it won't break with the smallest number of trial drops?


You might want to try this problem before you read the proof. On the next page I include my scratchwork.

\newpage
...

\newpage



\begin{thm}
If there are 2 cell phones and a 100 story building available then one can test dropping cell phones from 14 stories and find the maximum number of stories the cell phone can be dropped from without breaking.
\end{thm}

\begin{proof}
We will describe an algorithm for needing only to drop from 14 stories. Our first step will be to drop the first phone from the $14^{th}$ floor. If the phone breaks from this floor, we will test floors $1$ through $13$, in order, until the cell phone breaks, giving us up to a total of $14$ drops. If the phone does not break on a drop from the $14^{th}$ floor, we will drop the phone from the $27^{th}$ floor. If the phone breaks from the $27^{th}$ floor, we'll need to test floors $15$ through $26$. This is 12 more floors, in addition to the drop from the $14^{th}$ and the drop from the $27^{th}$ again giving us $14$ drops. 

We'll continue this process by dropping from floors $39, 50, 60, 69, 77, 84, 90, 95,$ and finally $99$. If we make it all the way to the $99^{th}$ floor then we will have done $11$ drops. In any other case, we will do $14$ total drops as seen by the following cases, which consider the first floor the phone breaks on from the list $39, 50, 60, 69, 77, 84, 90, 95$:
	\begin{itemize}
	\item The phone first breaks on the $39^{th}$ floor: In this case we test floors $14, 27, 39$ and floors $28-38$ giving $14$ total drops.
	\item The phone first breaks on the $50^{th}$ floor: In this case we test floors $14, 27, 39, 50$ and floors $40-49$ giving $14$ total drops.
	\item The phone first breaks on the $60^{th}$ floor: In this case we test floors $14, 27, 39, 50, 60$ and floors $51-59$ giving $14$ total drops.
	\item The phone first breaks on the $69^{th}$ floor: In this case we test floors $14, 27, 39, 50, 60, 69$ and floors $60-68$ giving $14$ total drops.
	\item The phone first breaks on the $77^{th}$ floor: In this case we test floors $14, 27, 39, 50, 60, 69, 77$ and floors $70-76$ giving $14$ total drops.
	\item The phone first breaks on the $84^{th}$ floor: In this case we test floors $14, 27, 39, 50, 60, 69, 77, 84$ and floors $78-83$ giving $14$ total drops.
	\item The phone first breaks on the $90^{th}$ floor: In this case we test floors $14, 27, 39, 50, 60, 69, 77, 84,90$ and floors $85-89$ giving $14$ total drops.
	\item The phone first breaks on the $95^{th}$ floor: In this case we test floors $14, 27, 39, 50, 60, 69, 77, 84,90,95$ and floors $91-94$ giving $14$ total drops.
	\end{itemize}

Thus we see that we can figure out the maximum floor where the phone breaks in $14$ total drops.
\end{proof}

\newpage

\section*{A Couple of Notes}

\begin{enumerate}
\item Note the proof does not claim that there isn't a better solution. Don't feel like you need to do every piece of a problem. Figure out something cool, and try to prove it.
\item On that note it's best to check your conjectures with me. I can help you if they are too hard, too obvious, too false, and also brainstorm proof ideas.
\item Remember that  a proof, at its heart, is just an explanation that other mathematicians believe. Don't feel like you need to adhere to strictly to a specific proof technique. Give a convincing explanation.
\item Pick a problem you like and start early so that this can be fun and not something you're worried about in the last week of the semester.
\end{enumerate}




\end{document}